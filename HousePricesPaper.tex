\documentclass[american,]{article}
\usepackage{lmodern}
\usepackage{amssymb,amsmath}
\usepackage{ifxetex,ifluatex}
\usepackage{fixltx2e} % provides \textsubscript
\ifnum 0\ifxetex 1\fi\ifluatex 1\fi=0 % if pdftex
  \usepackage[T1]{fontenc}
  \usepackage[utf8]{inputenc}
\else % if luatex or xelatex
  \ifxetex
    \usepackage{mathspec}
  \else
    \usepackage{fontspec}
  \fi
  \defaultfontfeatures{Ligatures=TeX,Scale=MatchLowercase}
\fi
% use upquote if available, for straight quotes in verbatim environments
\IfFileExists{upquote.sty}{\usepackage{upquote}}{}
% use microtype if available
\IfFileExists{microtype.sty}{%
\usepackage{microtype}
\UseMicrotypeSet[protrusion]{basicmath} % disable protrusion for tt fonts
}{}
\usepackage[margin=1in]{geometry}
\usepackage{hyperref}
\hypersetup{unicode=true,
            pdftitle={Predicting House Prices with a Linear Regression Model},
            pdfauthor={Kevin Thompson, Sterling Beason, \& Brandon Croom},
            pdfborder={0 0 0},
            breaklinks=true}
\urlstyle{same}  % don't use monospace font for urls
\ifnum 0\ifxetex 1\fi\ifluatex 1\fi=0 % if pdftex
  \usepackage[shorthands=off,main=american]{babel}
\else
  \usepackage{polyglossia}
  \setmainlanguage[variant=american]{english}
\fi
\usepackage{natbib}
\bibliographystyle{apalike}
\usepackage{longtable,booktabs}
\usepackage{graphicx,grffile}
\makeatletter
\def\maxwidth{\ifdim\Gin@nat@width>\linewidth\linewidth\else\Gin@nat@width\fi}
\def\maxheight{\ifdim\Gin@nat@height>\textheight\textheight\else\Gin@nat@height\fi}
\makeatother
% Scale images if necessary, so that they will not overflow the page
% margins by default, and it is still possible to overwrite the defaults
% using explicit options in \includegraphics[width, height, ...]{}
\setkeys{Gin}{width=\maxwidth,height=\maxheight,keepaspectratio}
\IfFileExists{parskip.sty}{%
\usepackage{parskip}
}{% else
\setlength{\parindent}{0pt}
\setlength{\parskip}{6pt plus 2pt minus 1pt}
}
\setlength{\emergencystretch}{3em}  % prevent overfull lines
\providecommand{\tightlist}{%
  \setlength{\itemsep}{0pt}\setlength{\parskip}{0pt}}
\setcounter{secnumdepth}{5}
% Redefines (sub)paragraphs to behave more like sections
\ifx\paragraph\undefined\else
\let\oldparagraph\paragraph
\renewcommand{\paragraph}[1]{\oldparagraph{#1}\mbox{}}
\fi
\ifx\subparagraph\undefined\else
\let\oldsubparagraph\subparagraph
\renewcommand{\subparagraph}[1]{\oldsubparagraph{#1}\mbox{}}
\fi

%%% Use protect on footnotes to avoid problems with footnotes in titles
\let\rmarkdownfootnote\footnote%
\def\footnote{\protect\rmarkdownfootnote}

%%% Change title format to be more compact
\usepackage{titling}

% Create subtitle command for use in maketitle
\providecommand{\subtitle}[1]{
  \posttitle{
    \begin{center}\large#1\end{center}
    }
}

\setlength{\droptitle}{-2em}

  \title{Predicting House Prices with a Linear Regression Model}
    \pretitle{\vspace{\droptitle}\centering\huge}
  \posttitle{\par}
    \author{Kevin Thompson, Sterling Beason, \& Brandon Croom}
    \preauthor{\centering\large\emph}
  \postauthor{\par}
      \predate{\centering\large\emph}
  \postdate{\par}
    \date{Data Science Program, Southern Methodist University, USA \break}

\usepackage{amsmath}
\usepackage[utf8]{inputenc}
\usepackage[T1]{fontenc}
\usepackage{setspace}
\onehalfspacing
\setcitestyle{round}
\newcommand\numberthis{\addtocounter{equation}{1}\tag{\theequation}}

\begin{document}
\maketitle
\begin{abstract}
Price prediction is pivotal for real estate. Homeowners on the sell-side
want to know when to sell, what to renovate, and how much profit they
can expect from their efforts. Homebuyers want to know whether they are
getting a fair price, where to look for homes in their budget, and the
various trade-offs that accompany a purchasing decision. Real estate
companies navigate both sides of real estate; hence, they too are a key
stakeholder. In the first part of our analysis, we estimate the
relationship between house prices, the square footage, and neighborhood
location in Ames, Iowa. In the second part of our analysis, we train a
linear regression model to predict house prices in Ames, Iowa.
\end{abstract}

\section{Introduction}\label{introduction}

\citet{Sleuth}

\section{Ames, Iowa Data}\label{ames-iowa-data}

The data used for this analysis described in the sections below comes
from the Kaggle Data set for House Prices (\citet{Kaggle2016}). The data
comes from the Ames, Iowa Housing dataset. This specific data set
contains a total of 2919 observations with 81 features. The data set is
broken up into a test data set, containing 1459 observations and 80
features (sale price is removed) and a training data set, containing
1460 observations with 81 features.

\citet{Kaggle2016}

\section{Analysis Question I}\label{analysis-question-i}

\citet{Sleuth}

\citet{Pearl2009}

\citet{Ruppert2015}

\section{Analysis Question II}\label{analysis-question-ii}

\citet{Hastie2009} \citet{Trefethen1997}

\section{Appendix}\label{appendix}

\renewcommand\refname{References}
\bibliography{HousePrices.bib}


\end{document}
